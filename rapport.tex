\documentclass[a4paper,10pt]{article}

\usepackage[french]{babel}
\usepackage[utf8]{inputenc}
\usepackage{lmodern}
\usepackage[T1]{fontenc}
\usepackage{hyperref}
\usepackage{graphicx}
\usepackage{url}


\title{MINI-PROJET AVEC L'ARCHITECTURE Y86}
\author{Nathan VANBESELAERE, Sacha DUPERRET}

\begin{document}

\maketitle

\vspace{30pt}

\begin{abstract}
Rapport détaillant le travail effectué en binôme pour adapter l'architecture Y86 aux consignes.
\end{abstract}

\vspace{30pt}

\tableofcontents

\newpage
\section{Exercice 1}
\subsection{Question 1}

Nous supprimons l'instruction :
\begin{verbatim}
    intsig MRMOVL                   'instructionSet.get("mrmovl").icode'
\end{verbatim}
Nous modifions également l'instruction set pour que RMMOVL ait un icode = 4 et ifun = 0, ainsi que MRMOVL ait un icode = 4 avec un ifun = 1.
Le code source Y86 compile bien, le code hexadécimal est bien de 40 et 41 pour RMMOVL (ifun = 0) et RMMOVL (ifun = 1) respectivement.

\subsection{Question 2}
Nous supprimons l'ensemble des occurences de MRMOVL.
Dans les cas où MRMOVL était dissocié de RMMOVL (dans le code HCL), nous ajoutons l'instruction :
\begin{verbatim}
    || icode == RMMOVL && ifun == 1
\end{verbatim}
permettant ainsi d'exécuter correctement le programme.

\section{Exercice 2}
\subsection{Question 1}
Nous modifions le instruction set en ajoutant STRGL avec un icode = 14 et un ifun = 0.
Nous testons cette nouvelle instruction en utilisant ce code :
\begin{verbatim}
    .pos 0
    init:   irmovl stack,%esp
            call   test
            halt

    test:   irmovl 3, %ecx
            strgl %ecx,%edx
            strgl %ecx,%ecx
            andl  %ecx,%ecx
            je     end
    end:    ret

            .align 4
    n:      .long  4
    s:      .long  0
    t:      .long  0x000a
            .long  0x00b0
            .long  0x0c00
            .long  0xd000

            .pos 0x100
    stack:
        .long 0

\end{verbatim}
L'exécution se déroulle conformément à nos attentes.

\subsection{Question 2}
Nous ajoutons l'instruction
\begin{verbatim}
    intsig STRGL                    'instructionSet.get("lodsl").icode'
\end{verbatim}
permettant de donner un icode à l'instruction STRGL.
Nous testons le code avec les même instructions que précédemment. Les valeurs des signaux et les opérations réalisées sont conformes à nos attentes.

\subsection{Question 3}
Suivant la même technique que pour la question 2 de l'execrcice 1, nous factorisons les cas commun à STRGL ifun = 0 || infun = 1.

\subsection{Question 4}
Nous codons un clone de strcpy en y86. Nous la testons dans le simulateur, avec le code ci-dessous :
\begin{verbatim}
            .pos 0
            irmovl stack,%esp

    main :  mrmovl size, %eax       #Taille du tableau
            pushl %eax

            irmovl t, %eax          #Source
            pushl %eax

            irmovl r, %eax          #Destination
            pushl %eax

            call strcpy

            iaddl 12, %esp
            halt


    strcpy: mrmovl 4(%esp),%edi
            mrmovl 8(%esp),%esi
            mrmovl 12(%esp),%ecx

    boucle: lodsl %eax
            stosl %eax
            isubl 1, %ecx
            jne boucle
            ret

            .pos 0x100
    size:   .long 5

    t:      .long 2                     #Source
            .long 3
            .long 5
            .long 7
            .long 11

    r:                              #Destination

            .pos 0x200
    stack:  .long 0

\end{verbatim}
Le comportement du code correspond aux attendus.

\section{Exercice 3}
\subsection{Question 1}
Nous ajoutons un icode de 15 avec ifun de 4 pour le code LOOP. Le code compile avec cette nouvelle instruction.

\subsection{Question 2}
Nous déclarons les signaux LOOP et RECX. Sous la forme :
\begin{verbatim}
    intsig LOOP                     'instructionSet.get("loop").icode'
    intsig RECX                     'registers.ecx'
\end{verbatim}


Dans new\_pc nous modifions la \textit{\#Taken branch: Use immediate value}.

\begin{verbatim}
    icode in { JXX } && Bch : valC;
    icode in { LOOP } && Bch : valC;
\end{verbatim}

Nous testons nos modifications avec ce code :

\begin{verbatim}
    .pos 0
irmovl t, %esi
irmovl r, %edi
mrmovl s, %ecx

boucle: lodsl %eax
        stosl %eax
        loop boucle
        halt

.pos 0x100
s:  .long 5
t:  .long 2
    .long 3
    .long 5
    .long 7
    .long 11
r:

\end{verbatim}
Le code compile et s'exécute sans erreur.

\section{Exercice 4}
\subsection{Question 1}
Nous ajoutons loope et loopne au jeu d'instruction (icode = 15, identique à celui de loop).
Pour les ifun, nous attibuons :
\begin{itemize}
    \item loop : 0
    \item loope : 3
    \item loopne : 4
\end{itemize}
Ce qui positionne ainsi le signal Bch conformément au comportement souhaité.

\subsection{Question 2}
Dans le code HCL, nous ajoutons le support de loope et loopne via la même procédure que dans les exercices précédents.

\subsection{Question 3}
Nous codons un clone de la fonction strncpy en Y86.
Pour cela, nous utilisons les fonctions codées précédemment.

Nous testons la fonction avec le code suivant :
\begin{verbatim}
    .pos 0
    irmovl stack,%esp

    main :  mrmovl limite, %eax         #Limite
            pushl %eax

            irmovl t, %eax              #Source
            pushl %eax

            irmovl r, %eax              #Destination
            pushl %eax

            call strcpy                 #Test premier cas
            call strcpy_bis             #Test deuxième cas
            iaddl 12, %esp
            halt


    strcpy: mrmovl 4(%esp),%edi         #Test premier cas
            mrmovl 8(%esp),%esi
            mrmovl 12(%esp),%ecx

    boucle: lodsl %eax
            stosl %eax
            isubl 0, %eax               #On vérifie si le tableau est vide
            jne nonvide                 #Si le tableau n'est pas encore vide
            rmmovl %eax,-4(%edi)        #Si le tableau est vide,
                                        #on ajoute un 0 à la suite de notre tableau
                                        #de destination

    nonvide:loopne boucle
            ret


    strcpy_bis: irmovl b, %esi          #Test deuxième cas
                mrmovl 4(%esp),%edi
                mrmovl 12(%esp),%ecx
                jmp boucle              #Renvoi à notre boucle avec notre 2nd tableau


    .pos 0x100
    limite:  .long 4                    #Limite

    t:  .long 2                         #Tableau premier cas
        .long 3

    .pos 0x120
    b:  .long 2                         #Tableau deuxième cas
        .long 3
        .long 5
        .long 7
        .long 11
    r:                                  #Destination

    .pos 0x200
    stack:  .long 0

\end{verbatim}
Notre clone de la fonction strncpy fonctionne correctement dans les deux différents cas.

\end{document}

